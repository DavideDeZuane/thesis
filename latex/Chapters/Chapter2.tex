% Chapter Template

\newcommand{\drawkey}[3]{
    \draw[line width=0.07cm, draw=#2] (#1) circle [radius=0.15cm];
    \draw[line width=0.07cm, draw=#2] (#1 -0.15) -- ++(0,-0.4);
    \draw[line width=0.07cm, draw=#2] (#1 -0.35) -- ++(-0.2,0);
    \draw[line width=0.07cm, draw=#2] (#1 -0.51) -- ++(-0.2,0);
    \node[below, text=#2] at (#1 -0.6) {#3}
}

% Definizione del comando per disegnare la parentesi
\newcommand{\drawcurlybrace}[3]{% posizione finale, posizione iniziale, testo sopra
    \draw [decorate,decoration={brace,amplitude=10pt,mirror},xshift=-4pt,yshift=0pt]
    (#1) -- (#2) node [black,midway,xshift=-2cm] {};
}

\chapter{IKEv2}
\label{Chapter2} 

\section{Pre-Quantum}

Andiamo a vedere quello che era il funzionamento di IKEv2 nel caso di crittografia classica.
IkEv2 funziona sul principio di \textit{exchange} ovvero di richiesta e risposta, in Fig. \ref{fig:ike_exchange} sono riportate le varie interazioni tra 
initiator e responder volte ad instaurare una Security Association.

\begin{figure}[htbp]
    \centering
    \begin{tikzpicture}[node distance=1.5cm]
        % Initiator and Responder
        \node (entity1) [draw=red, rectangle] {Initiator (\textit{i})};
        \node (entity2) [draw=blue, rectangle, right=of entity1, xshift=3cm] {Responder (\textit{r})};
        \draw (entity1) -- ++(0,-6.7) coordinate (vertical1);
        \draw (entity2) -- ++(0,-6.7);
        % IKE_INIT_SA
        \draw[-stealth] (entity1) ++(0,-1) -- (entity2 |-,-1) node[midway, above, text=red, font=\footnotesize] {$SA_{i1}, KE_i, N_i$};
        \draw[stealth-] (entity1) ++(0,-2) -- (entity2 |-,-2) node[midway, above, text=blue, font=\footnotesize] {$SA_{r1},KE_r,N_r$};
        % IKE_AUTH
        \draw[-stealth] (entity1) ++(0,-3.5) -- (entity2 |-,-3.5) node[midway, above, text=red, font=\footnotesize] {$SK\{ID_i, CERT, \{AUTH\}^s_{pk}\}$};
        \draw[stealth-] (entity1) ++(0,-4.5) -- (entity2 |-,-4.5) node[midway, above, text=blue, font=\footnotesize] {$SK\{ID_r, CERT, \{AUTH\}^s_{pk}\}$};
        % CHILD_SA 
        \node at (-1.7,-6) {\inlinecode{CHILD\_SA}};
        \node at (-2,-1.5) {\inlinecode{IKE\_SA\_INIT}};
        \node at (-1.7,-4) {\inlinecode{IKE\_AUTH}};
        \draw (entity1) ++(0.2,-6) ellipse (0.2cm and 0.2cm);
        \draw [-] (0.2,-6.2) -- (6.65,-6.2);
        \draw [-] (0.2,-5.8) -- (6.65,-5.8) node[midway, above, font=\footnotesize]{$SK_d, N_i, N_r$};
        \draw (entity2) ++(-0.2,-6) ellipse (0.2cm and 0.2cm);
        % Key for encryption and authtenticatoin
        \drawkey{-1, -2.3}{red}{$SK_{ei}$};
        \drawkey{-1.8, -2.3}{red}{$SK_{ai}$};
        \drawkey{8, -2.3}{blue}{$SK_{ar}$};
        \drawkey{8.8, -2.3}{blue}{$SK_{er}$};
        % Group
        \drawcurlybrace{0,-1}{0, -2}{};
        \drawcurlybrace{0,-3.5}{0, -4.5}{};
    \end{tikzpicture}
    \label{fig:ike_exchange}
    \caption{Rappresentazioni delle fasi di IKEv2}
\end{figure}


\subsection{\textbf{IKE\_SA\_INIT}}

Lo scopo di questa prima fase è quello di creare una \textbf{IKE SA}, che consenta di rendere sicure i successivi scambi di dati al fine di realizzare una \textbf{IPsec SA}.
Dunque funge da apripista al fine di stabilire quelli che sono i parametri di sicurezza al fine di avere una comunicazione sicura. Per questo motivo in questo scambio i peer 
si scambiano le seguenti informazioni:

\begin{table}[htbp]
    \centering
    \caption{Tabella dei parametri e delle descrizioni}
    \begin{tabular}{ll}
        \toprule
        \textbf{Parametro} & \textbf{Descrizione} \\
        \midrule
        \textit{SA} & Security Association, vengono negoziati i parametri per la SA\\
        \textit{KE} & Key Exchange, e nel caso classico è l'esponente DH \\
        \textit{N} & Nonce \\
    \bottomrule
    \end{tabular}
\end{table}

\noindent
Al termine di questo scambio i due peer ottengono il \textit{DH Shared Secret} (indicato con $g^{ir}$), il quale insisme ai nonce, consentirà di ottenere 
quelli che sono i parametri di sicurezza della $IKE SA$ al fine di instauare un canale sicuro, per approfondimenti in \href{AppendixA}{appendice}.

% Tabella che riporta le varie funzioni delle chiavi

\subsection{IKE\_AUTH}

Il risultato della fase precedente è un canale sicuro su cui comunicare, in quanto è cifrato e utenticato. Si questo hanno luogo gli scambi per instaurare la IPsec SA.
In questa fase i nodi si autenticano mutuamente:

\begin{table}[htbp]
    \centering
    \caption{Tabella dei parametri e delle descrizioni}
    \begin{tabular}{ll}
        \toprule
        \textbf{Parametro} & \textbf{Descrizione} \\
        \midrule
        \textit{AUTH} & Payload che deve essere firmato affinchè ci sia autenticazione \\
        \textit{CERT} & Si allega il certificato digitale per la chiave pubblica \\
        \textit{CERTQ} & Si fa richiesta al peer di fornire il certificato \\
    \bottomrule
    \end{tabular}
\end{table}

Tutto il contenuto appena descritto è protetto mediante le chiavi segrete di quella direzione. Ciò è indicato mediante la notazione $SK\{...\}$
La modalità di autenticazione può essere: PSK, EAP oppure mediante chiave pubblica.

\subsection{CHILD\_SA}


\section{Problemi}

IKEv2 utilizza come porotocollo a livello trasporto UDP per per inoltrare i propri messaggi. La maggior parte dei messaggi che i peer si scambiano hanno dimensioni relativamente piccole e
quindi che non eccedono l'\textbf{MTU} di un pacchetto IP, tuttavia abbiamo degli scambi che richiedono un trasferimento di dati abbastanza grandi.

Per esempio nel caso di autenticazione tramite pubkey nella fase di \inlinecode{IKE\_AUTH} è necessario trasferire il proprio certificato che in base allo schema di firma utilizzato
può arrivare anche a diversi Kbyte di dimensione. In questi casi si verifica la frammentazione a livello IP.

%Immagine frammentazione

\begin{figure}[htbp]
    \centering
    \begin{tikzpicture}[node distance=2cm,>=Latex]
        % Nodi
        \node (initiator) [client] {Initiator};
        \node (device) [router, right=of initiator] {Device};
        \node (responder) [client, right=of device] {Responder};
        \node (message) [messageclosed, fill=orange!40, minimum size=0.6cm] at (1.9,0.5) {};
        \node (message2) [messageclosed, fill=gray!40, minimum size=0.6cm] at (1.2,0.5) {};
        \node (drop) [messageclosed, fill=gray!40, minimum size=0.6cm, , font=\small, text=red] at (3,1) {};
        \node (message3) [messageclosed, fill=orange!40, minimum size=0.6cm, , font=\small, text=red] at (4.5,0.5) {};

        \draw[-] (initiator.east) -- (device.west);
        \draw[-] (device.east) -- (responder.west);

    \end{tikzpicture}
    \vspace*{1cm}
    \caption{Drop Pacchetti}
    \label{fig:cgnatdrop}
\end{figure}

Diversi test hanno mostrato che nel caso in cui i peer si trovino in presenza di CGNAT potrebbero non istausarsi le SA. Questo è dovuto al fatto che i device degli ISP non
consentono ai frammenti IP di passare attravers di loro, ovvero scartano i pacchetti e di conseguenza bloccano le comunicazioni IKE.
Questo è riportato schematicamente in Fig. \ref{fig:cgnatdrop}.
Questo drop dei pacchetti avviene perchè esistono numerosi vettori di attacco che fanno affidamento sulla frammentazione IP, per questo motivo gli ISP operano un filtro su questa tipologia di pacchetti.
Ance se in teoria uno dei requisiti del CGNAT definito dagli RFC è proprio consentire la frammentazione.

Per risolvere questa problematica e dunque consentire il passaggio dei messaggi attraverso i dispositivi di rete che non consentono il passaggio degli IP fragment attraverso 
di loro nell' RFC 7283 viene introdotta la IKEv2 \textit{Message Fragmentation}. In cui la frammentazione dei messaggi è gestita direttamente da parte di chi implementa IKEv2

\subsection{\inlinecode{IKE\_INTERMEDIATE}}

Per evitare che nel trasferimento di grandi dati ciò avvenga viene introdotto uno scambio aggiuntivo. Questo scambio è introdotto per quei casi in cui la dimensione dei dati
da trasferire ecceda la dimensione massima che causerebbe la frammentazione IP. Questo scambio va fatto dopo la \inlinecode{IKE\_INIT\_SA} e prima della \inlinecode{IKE\_AUTH} in questo
modo è sia autenticato che cifrato tramite le chiavi negoziate dal primo scambio.

\begin{figure}[htbp]
    \centering
    \begin{tikzpicture}[node distance=1.5cm]
        % Initiator and Responder
        \node (entity1) [draw=red, rectangle] {Initiator (\textit{i})};
        \node (entity2) [draw=blue, rectangle, right=of entity1, xshift=3cm] {Responder (\textit{r})};
        \draw (entity1) -- ++(0,-4.5) coordinate (vertical1);
        \draw (entity2) -- ++(0,-4.5);
        % IKE_AUTH
        \draw[stealth-stealth] (entity1) ++(0,-1.5) -- (entity2 |-,-1.5) node[midway, above] {\inlinecode{IKE\_SA\_INIT}};
        \draw[stealth-stealth] (entity1) ++(0,-2.5) -- (entity2 |-,-2.5) node[midway, above] {[\inlinecode{IKE\_INTERMEDIATE}]};
        \draw[stealth-stealth] (entity1) ++(0,-3.5) -- (entity2 |-,-3.5) node[midway, above] {\inlinecode{IKE\_AUTH}};
    \end{tikzpicture}
    \caption{Scambio nuovo}
    \label{fig:ikeintermediate}
\end{figure}

Questo scambio è posizionato qui in quanto nella \inlinecode{IKE\_SA\_INIT} per motivi di sicurezza non è possibile applicare la frammentazione.
Di solito i messaggi sono piccoli abbastanza da non causare la frammentazione IP, tuttavia questo potrebbe cambiare se si utilizzano scambi di chiave QC-resistant; in quanto
hanno chiavi pubbliche larghe e che quindi causerebbero frammentazione IP.

Per questo viene aggiunto questo scambio che viene utilizzato per trasferire grandi quantità di dati.

L'utilizzo principale di questo scambio è quello di trasferire le chiavi pubbliche QC-resistant, tuttavia in generale può essere utilizzzato per trasferire qualsiasi tipologia di dato.
Quindi il principale utilizzo è quello di fare un \textbf{enforcing} delle chiavi negoziate tramite DH al fine di renderle QC-resistant. Infatti se durante questo scambio si scambiano 
altre chiavi allora le coppie $\{SK_{a[i/r]}, SK_{e[i/r]}\}$ vengono aggiornate.

Permette di realizzare Multiple Key Exchange
Gli scambi di chiave aggiuntivi vengono aggiunti alla proposal tramite \inlinecode{PQ\_KEM\_1}



Lo scambio \inlinecode{IKE\_FOLLOWUP\_KE} è introdotto specificatamente per trasferire dati sulla chiavi addizionali da realizzare in una CHILD SA.
In questo caso le chiavi aggiuntive vengono utilizzate per aggiornare il KEYMAT



\begin{itemize}
    \item flag \inlinecode{IKE\_FRAGMENTATION\_SUPPORT}: il peer dice di supportare la frammentazione IKEv2, affinchè venga utilizzata entrambi i peer devono supportarla.
    \item flag \inlinecode{INTERMEDIATE\_EXCHANGE\_SUPPORT}: il peer dice di supportare gli scambi intermedi
\end{itemize}

Una volta terminati gli scambi, per proteggere lo scambio \inlinecode{IKE\_AUTH} e gli scambi successivi vengno utilizzata le ultime chiavi calcolate
Dato che i dati trasferiti in questi scambi aggiuntivi vanno autenticati si aggiungono all'\inlinecode{AUTH} payload che poi andrà 



Il supporto per lo scambio aggiuntivo viene comunicato aggiungendo all'interno
dell \inlinecode{IKE\_SA\_INIT} il flag \inlinecode{IKE\_INT\_SUP} (che sta per
Intermediate Exchange Support).
Se anche il responder lo supporta lo includerà nel messaggio di risposta dello
scambio.




Considerazioni, L'IKE fragmentation viene introdotta a causa del NAT tuttavia nel nostro caso di satelliti non ha senso utilizzarla in quanto non credo che si utilizzi
il NAT soprattutto perchè introduce ritardi dovuti alla traduzione degli indirizzi



\section{Post-Quantum}

Un solo KEM con Kyber L1 usando come suite AES\_GCM has vabene come certificato dilithium L1

Nel KEM quanti cifrano?

Cioè l'initiator manda il certificato e poi il responder cifra 