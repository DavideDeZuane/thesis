% Chapter Template
\newcommand{\drawkey}[3]{
    \draw[line width=0.07cm, draw=#2] (#1) circle [radius=0.15cm];
    \draw[line width=0.07cm, draw=#2] (#1 -0.15) -- ++(0,-0.4);
    \draw[line width=0.07cm, draw=#2] (#1 -0.35) -- ++(-0.2,0);
    \draw[line width=0.07cm, draw=#2] (#1 -0.51) -- ++(-0.2,0);
    \node[below, text=#2] at (#1 -0.6) {#3}
}

% Definizione del comando per disegnare la parentesi
\newcommand{\drawcurlybrace}[3]{% posizione finale, posizione iniziale, testo sopra
    \draw [decorate,decoration={brace,amplitude=10pt,mirror},xshift=-4pt,yshift=0pt]
    (#1) -- (#2) node [black,midway,xshift=-2cm] {};
}

\chapter{Fondamenti di Comunicazioni Sicure}
\label{Capitolo 2} 

In questo capitolo andiamo a trattare quelle che sono le problematiche riguardanti le comunicazioni sicure.
E' possibile realizzare comunicazioni sicure grazie a quelli che cono i crittosistemi, andiamo a vedere come questi vengono utilizzati nelle reti 
di computer per garantire la sicurezza delle comunicazioni.

\section{Teoria}

Introduciamo quelli che sono gli strumenti fondamentali per la crittografia.

\subsection{Hash Function}

\subsection{Schemi crittografici}
Uno schema di cifratura è un insieme di algoritmi e funzioni che definisce come trasformare un messaggio in chiaro (plaintext) in un messaggio cifrato (ciphertext) e viceversa, al fine di garantire la confidenzialità e la sicurezza delle comunicazioni.
Formalmente possiamo rappresentarlo come una quintupla: 

\begin{center}
    \((\mathcal{P}, \mathcal{C}, \mathcal{K}, E, D)\)
\end{center}

\begin{itemize}
    \item \(\mathcal{P}\): Insieme dei messaggi in chiaro (plaintext).
    \item \(\mathcal{C}\): Insieme dei messaggi cifrati (ciphertext).
    \item \(\mathcal{K}\): Insieme delle chiavi utilizzate per la cifratura e decifratura,\textit{key space}.
    \item \(E: \mathcal{K} \times \mathcal{P} \to \mathcal{C}\): Funzione di cifratura.
    \item \(D: \mathcal{K} \times \mathcal{C} \to \mathcal{P}\): Funzione di decifratura.
\end{itemize}

\noindent
Deve esistere una relazione inversa tra le operazioni di cifratura e decifratura:
\begin{equation}
D(k, E(k, m)) = m \quad \forall m \in \mathcal{P}, \, k \in \mathcal{K}
\end{equation}

\noindent
Questa definizione di schema di cifratura è generale, tuttavia in base alla definizione e all'uso delle chiavi, è possibile distinguere tra schemi di cifratura simmetrica e asimmetrica.

\subsubsection{Simmetrici}

In uno schema simmetrico si utilizza la stessa chiave sia per le operazioni di cifratura che di decifratura. 
Significa che le due parti coinvolte nella comunicazione, ovvero che si vogliono scambiare un messaggio devono possedere la stessa chiave segreta (PSK).

Alcuni esempi di crittosistemi simmetrici sono DES, AES,...


\subsubsection{Asimmetrici}

In uno schema di cifratura asimmetrica, vengono utilizzate due chiavi distinte: una chiave pubblica e una chiave privata.

Lo spazio delle chiavi \(\mathcal{K}\) è costituito da una coppia di chiavi \((k_{\text{pub}}, k_{\text{priv}})\), dove:

\begin{itemize}
    \item \(k_{\text{pub}}\) è la chiave pubblica (usata per cifrare) 
    \item \(k_{\text{priv}}\) è la chiave privata (usata per decifrare)    
\end{itemize}

\noindent
Le due chiavi sono matematicamente legate, ma è computazionalmente difficile ottenere la chiave privata a partire da quella pubblica (questa è la base della sicurezza).
Quindi le due funzioni si riscrivono come:

\begin{equation}
    E: \mathcal{K}_{\text{pub}} \times \mathcal{P} \to \mathcal{C}
\end{equation}
\begin{equation}
    D: \mathcal{K}_{\text{priv}} \times \mathcal{C} \to \mathcal{P}
\end{equation}


\begin{figure}[htbp]
    \centering
    \begin{tikzpicture}[node distance=1.5cm]
        \node (entity1) [draw=green, rectangle] {Alice};
        \drawkey{-0.8, -0.8}{black}{$K_{pri}$};
        \drawkey{-1.6, -0.8}{black}{$K_{pub}$};
        \node (entity2) [draw=blue, rectangle, right=of entity1, xshift=3cm] {Bob};
        \draw (entity1) -- ++(0,-5) coordinate (vertical1);
        \draw (entity2) -- ++(0,-5);
        \draw[-stealth] (entity1) ++(0,-1) -- (entity2 |-,-1) node[midway, above, text=black, font=\footnotesize] {$K_{pub}$};
        \draw[stealth-] (entity2) ++(0,-2) -- (entity1 |-,-2) node[midway, above, text=black, font=\footnotesize] {$E(K_{pub},m)$};
    \end{tikzpicture}
    \label{}
    \caption{}
\end{figure}

Successivamente andremo nel dettaglio di quelli che sono i principali utilizzi dei crittosistemi asimmetrici.


\subsubsection*{Scambio di Chiavi}
\subsubsection*{Cifratura Asimmetrica}
\subsubsection*{Firma Digitale}

Alcuni esempi di quelli classici sono: RSA, Diffie-Hellman,..

\subsection{Confronto}

Vediamo le principali differenze tra i due 

Distribuzione delle chiavi KDC e PKI.

\subsection{Sicurezza}

Il \textbf{security level} è una misura della forza che una primitiva crittografica raggiunge rispetto ad attacchi.
Solitamente viene espresso come un numero di “bit di sicurezza”, dove $n$-bit di sicurezza significa che l'attaccante dovrebbe eseguire $2*n$ operazioni per romperlo.

Per i cifrari simmetrici, il \textit{livello di sicurezza} è pari alla dimensione del key-space. Mentre la sicurezza degli algoritmi asimmetrici si basa su problemi matematici che sono efficienti da calcolare in una direzione, ma inefficienti da invertire da parte dell'attaccante.
Tuttavia, gli attacchi contro gli attuali sistemi a chiave pubblica sono sempre più veloci della ricerca a forza bruta dello spazio delle chiavi.

Il NIST (National Institute of Standards and Technology) ha introdotto livelli di sicurezza per gli algoritmi di cifratura asimmetrica e post-quantistica come parte della sua iniziativa per standardizzare algoritmi che resistano anche ai computer quantistici.
I livelli sono definiti in \textit{Tabella \ref{tab:security-levels}}.

% NIST Security Level Table
\renewcommand{\arraystretch}{1.3} 
\begin{table}[ht]
    \centering
    \begin{tabular}{>{\centering\arraybackslash}m{3cm}p{10cm}}
        \hline
        \textbf{Security Level} & \textbf{Descrizione} \\
        \hline
        \textbf{Livello 1} & Sicurezza equivalente alla cifratura simmetrica con chiavi da 128 bit, come AES-128.\\
        \textbf{Livello 2} & Sicurezza equivalente ad attacchi contro SHA-256, con complessità circa pari a 128 bit. Leggermente più sicuro del livello 1. \\
        \textbf{Livello 3} & Sicurezza equivalente alla cifratura simmetrica con chiavi da 192 bit, come AES-192. \\
        \textbf{Livello 4} & Sicurezza equivalente ad attacchi contro SHA-384. Leggermente più sicuro del Livello 3. \\
        \textbf{Livello 5} & Sicurezza equivalente alla cifratura simmetrica con chiavi da 256 bit, come AES-256. \\
        \hline
    \end{tabular}
    \caption{Security Levels definiti dal NIST}
    \label{tab:security-levels}
\end{table}


\section{Applicazioni}

Le principali applicazioni crittografiche utilizzate oggi in contesti reali includono SSL/TLS, PGP, IPsec e tante altre. 
Tali applicazioni si basano su crittosistemi che combinano algoritmi di cifratura, autenticazione e gestione delle chiavi per garantire la riservatezza e l'integrità dei dati scambiati.


Nel contesto delle comunicazioni digitali, la crittografia gioca un ruolo fondamentale per garantire la sicurezza dei dati scambiati tra entità remote. 
Uno dei principali modelli di riferimento per la trasmissione di dati su Internet è il modello TCP/IP, che suddivide il processo di comunicazione in diversi livelli, ciascuno con funzioni specifiche.


Così come per le comunicaizoni di tutti i giorni, anche per quelle che avvengono attravers Internet vogliamo rispettare quelli che sono i requisiti di
sicurezza informatica ovvero garantire:

\begin{itemize}
    \item Confidenzialità 
    \item Integrità
    \item Autenticazione
\end{itemize}

Le comunicazioni seguono il modello ISO/OSI per questo possiamo applicarla a diversi livelli
Come viene applicata la crittografia nelle comunicazioni, quindi i vari metodi con cui la sicurezza è applicata ai vari livelli dello stack network.
Si parla di comunicazioni sicure ma a livello pratico queste come vengono realizzate, esiste una suitte di protocolli che consente di ottenerle.



\subsection{IPsec}

Facciamo un focus sulla sicurezza a livello IP, quindi introduciamo la suite di protocolli IPsec
Le comunicazioni sicure si costruiscono sopra un concetto fondamentale noto come Security Association (SA). 
Le comunicazioni sono protette attraverso l'utilizzo di tre tecniche crittografiche:

\begin{itemize}
    \item Autenticazione
    \item Integrità
    \item Cifratura
\end{itemize}


La comunicazioni possono rendersi sicure a diversi livelli dello stack network, tuttavia farla a livello IP ha il grande vantaggio di progettere le comunicazioni 
per tutte le applicazioni, senza andarne a modificare ciascuna suingolarmente. 


Il protocollo IKEv2, definito nell'\texttt{RFC 7296}, è un protocollo di rete che è responsabile della negoziazione di chiavi crittografiche e dei parametri di sicurezza tra due dispositivi, solitamente chiamati peer.

\begin{figure}[htbp]
    \centering
    \begin{tikzpicture}[node distance=1.5cm]
        % Initiator and Responder
        \node (entity1) [draw=red, rectangle] {Initiator (\textit{i})};
        \node (entity2) [draw=blue, rectangle, right=of entity1, xshift=3cm] {Responder (\textit{r})};
        \draw (entity1) -- ++(0,-6.7) coordinate (vertical1);
        \draw (entity2) -- ++(0,-6.7);
        % IKE_INIT_SA
        \draw[-stealth] (entity1) ++(0,-1) -- (entity2 |-,-1) node[midway, above, text=red, font=\footnotesize] {$SA_{i1}, KE_i, N_i$};
        \draw[stealth-] (entity1) ++(0,-2) -- (entity2 |-,-2) node[midway, above, text=blue, font=\footnotesize] {$SA_{r1},KE_r,N_r$};
        % IKE_AUTH
        \draw[-stealth] (entity1) ++(0,-3.5) -- (entity2 |-,-3.5) node[midway, above, text=red, font=\footnotesize] {$SK\{ID_i, CERT, \{AUTH\}^s_{pk}\}$};
        \draw[stealth-] (entity1) ++(0,-4.5) -- (entity2 |-,-4.5) node[midway, above, text=blue, font=\footnotesize] {$SK\{ID_r, CERT, \{AUTH\}^s_{pk}\}$};
        % CHILD_SA 
        \node at (-1.7,-6) {\inlinecode{CHILD\_SA}};
        \node at (-2,-1.5) {\inlinecode{IKE\_SA\_INIT}};
        \node at (-1.7,-4) {\inlinecode{IKE\_AUTH}};
        \draw (entity1) ++(0.2,-6) ellipse (0.2cm and 0.2cm);
        \draw [-] (0.2,-6.2) -- (6.65,-6.2);
        \draw [-] (0.2,-5.8) -- (6.65,-5.8) node[midway, above, font=\footnotesize]{$SK_d, N_i, N_r$};
        \draw (entity2) ++(-0.2,-6) ellipse (0.2cm and 0.2cm);
        % Key for encryption and authtenticatoin
        \drawkey{-1, -2.3}{red}{$SK_{ei}$};
        \drawkey{-1.8, -2.3}{red}{$SK_{ai}$};
        \drawkey{8, -2.3}{blue}{$SK_{ar}$};
        \drawkey{8.8, -2.3}{blue}{$SK_{er}$};
        % Group
        \drawcurlybrace{0,-1}{0, -2}{};
        \drawcurlybrace{0,-3.5}{0, -4.5}{};
    \end{tikzpicture}
    \label{fig:ike_exchange}
    \caption{Fasi di Negoziazione del Protocollo IKEv2}
\end{figure}

Lo schema generale di funzionamento del protocollo al termine del quale i due hanno stabilito una SA è riportato in \textit{Fig. \ref{fig:ike_exchange}}.

\subsection{\textbf{IKE\_SA\_INIT}}

Lo scopo di questa prima fase è quello di creare una \textbf{IKE SA}, che consenta di rendere sicure i successivi scambi di dati al fine di realizzare una \textbf{IPsec SA}.
Dunque funge da apripista al fine di stabilire quelli che sono i parametri di sicurezza al fine di avere una comunicazione sicura. Per questo motivo in questo scambio i peer 
si scambiano le seguenti informazioni:

\begin{table}[htbp]
    \centering
    \caption{Tabella dei parametri e delle descrizioni}
    \begin{tabular}{ll}
        \toprule
        \textbf{Parametro} & \textbf{Descrizione} \\
        \midrule
        \textit{SA} & Security Association, vengono negoziati i parametri per la SA\\
        \textit{KE} & Key Exchange, e nel caso classico è l'esponente DH \\
        \textit{N} & Nonce \\
    \bottomrule
    \end{tabular}
\end{table}

\noindent
Al termine di questo scambio i due peer ottengono il \textit{DH Shared Secret} (indicato con $g^{ir}$), il quale insisme ai nonce, consentirà di ottenere 
quelli che sono i parametri di sicurezza della $IKE SA$ al fine di instauare un canale sicuro, per approfondimenti in \href{AppendixA}{appendice}.

% Tabella che riporta le varie funzioni delle chiavi

\subsection{IKE\_AUTH}

Il risultato della fase precedente è un canale sicuro su cui comunicare, in quanto è cifrato e utenticato. Si questo hanno luogo gli scambi per instaurare la IPsec SA.
In questa fase i nodi si autenticano mutuamente:

\begin{table}[htbp]
    \centering
    \caption{Tabella dei parametri e delle descrizioni}
    \begin{tabular}{ll}
        \toprule
        \textbf{Parametro} & \textbf{Descrizione} \\
        \midrule
        \textit{AUTH} & Payload che deve essere firmato affinchè ci sia autenticazione \\
        \textit{CERT} & Si allega il certificato digitale per la chiave pubblica \\
        \textit{CERTQ} & Si fa richiesta al peer di fornire il certificato \\
    \bottomrule
    \end{tabular}
\end{table}

Tutto il contenuto appena descritto è protetto mediante le chiavi segrete di quella direzione. Ciò è indicato mediante la notazione $SK\{...\}$
La modalità di autenticazione può essere: PSK, EAP oppure mediante chiave pubblica.

\subsection{CHILD\_SA}


\section{Problemi}

IKEv2 utilizza come porotocollo a livello trasporto UDP per per inoltrare i propri messaggi. La maggior parte dei messaggi che i peer si scambiano hanno dimensioni relativamente piccole e
quindi che non eccedono l'\textbf{MTU} di un pacchetto IP, tuttavia abbiamo degli scambi che richiedono un trasferimento di dati abbastanza grandi.

Per esempio nel caso di autenticazione tramite pubkey nella fase di \inlinecode{IKE\_AUTH} è necessario trasferire il proprio certificato che in base allo schema di firma utilizzato
può arrivare anche a diversi Kbyte di dimensione. In questi casi si verifica la frammentazione a livello IP.

%Immagine frammentazione

\begin{figure}[htbp]
    \centering
    \begin{tikzpicture}[node distance=2cm,>=Latex]
        % Nodi
        \node (initiator) [client] {Initiator};
        \node (device) [router, right=of initiator] {Device};
        \node (responder) [client, right=of device] {Responder};
        \node (message) [messageclosed, fill=orange!40, minimum size=0.6cm] at (1.9,0.5) {};
        \node (message2) [messageclosed, fill=gray!40, minimum size=0.6cm] at (1.2,0.5) {};
        \node (drop) [messageclosed, fill=gray!40, minimum size=0.6cm, , font=\small, text=red] at (3,1) {};
        \node (message3) [messageclosed, fill=orange!40, minimum size=0.6cm, , font=\small, text=red] at (4.5,0.5) {};

        \draw[-] (initiator.east) -- (device.west);
        \draw[-] (device.east) -- (responder.west);

    \end{tikzpicture}
    \vspace*{1cm}
    \caption{Drop Pacchetti}
    \label{fig:cgnatdrop}
\end{figure}

Diversi test hanno mostrato che nel caso in cui i peer si trovino in presenza di CGNAT potrebbero non istausarsi le SA. Questo è dovuto al fatto che i device degli ISP non
consentono ai frammenti IP di passare attravers di loro, ovvero scartano i pacchetti e di conseguenza bloccano le comunicazioni IKE.
Questo è riportato schematicamente in Fig. \ref{fig:cgnatdrop}.
Questo drop dei pacchetti avviene perchè esistono numerosi vettori di attacco che fanno affidamento sulla frammentazione IP, per questo motivo gli ISP operano un filtro su questa tipologia di pacchetti.
Ance se in teoria uno dei requisiti del CGNAT definito dagli RFC è proprio consentire la frammentazione.

Per risolvere questa problematica e dunque consentire il passaggio dei messaggi attraverso i dispositivi di rete che non consentono il passaggio degli IP fragment attraverso 
di loro nell' RFC 7283 viene introdotta la IKEv2 \textit{Message Fragmentation}. In cui la frammentazione dei messaggi è gestita direttamente da parte di chi implementa IKEv2

\subsection{\inlinecode{IKE\_INTERMEDIATE}}

Per evitare che nel trasferimento di grandi dati ciò avvenga viene introdotto uno scambio aggiuntivo. Questo scambio è introdotto per quei casi in cui la dimensione dei dati
da trasferire ecceda la dimensione massima che causerebbe la frammentazione IP. Questo scambio va fatto dopo la \inlinecode{IKE\_INIT\_SA} e prima della \inlinecode{IKE\_AUTH} in questo
modo è sia autenticato che cifrato tramite le chiavi negoziate dal primo scambio.

\begin{figure}[htbp]
    \centering
    \begin{tikzpicture}[node distance=1.5cm]
        % Initiator and Responder
        \node (entity1) [draw=red, rectangle] {Initiator (\textit{i})};
        \node (entity2) [draw=blue, rectangle, right=of entity1, xshift=3cm] {Responder (\textit{r})};
        \draw (entity1) -- ++(0,-4.5) coordinate (vertical1);
        \draw (entity2) -- ++(0,-4.5);
        % IKE_AUTH
        \draw[stealth-stealth] (entity1) ++(0,-1.5) -- (entity2 |-,-1.5) node[midway, above] {\inlinecode{IKE\_SA\_INIT}};
        \draw[stealth-stealth] (entity1) ++(0,-2.5) -- (entity2 |-,-2.5) node[midway, above] {[\inlinecode{IKE\_INTERMEDIATE}]};
        \draw[stealth-stealth] (entity1) ++(0,-3.5) -- (entity2 |-,-3.5) node[midway, above] {\inlinecode{IKE\_AUTH}};
    \end{tikzpicture}
    \caption{Scambio nuovo}
    \label{fig:ikeintermediate}
\end{figure}

Questo scambio è posizionato qui in quanto nella \inlinecode{IKE\_SA\_INIT} per motivi di sicurezza non è possibile applicare la frammentazione.
Di solito i messaggi sono piccoli abbastanza da non causare la frammentazione IP, tuttavia questo potrebbe cambiare se si utilizzano scambi di chiave QC-resistant; in quanto
hanno chiavi pubbliche larghe e che quindi causerebbero frammentazione IP.

Per questo viene aggiunto questo scambio che viene utilizzato per trasferire grandi quantità di dati.

L'utilizzo principale di questo scambio è quello di trasferire le chiavi pubbliche QC-resistant, tuttavia in generale può essere utilizzzato per trasferire qualsiasi tipologia di dato.
Quindi il principale utilizzo è quello di fare un \textbf{enforcing} delle chiavi negoziate tramite DH al fine di renderle QC-resistant. Infatti se durante questo scambio si scambiano 
altre chiavi allora le coppie $\{SK_{a[i/r]}, SK_{e[i/r]}\}$ vengono aggiornate.

Permette di realizzare Multiple Key Exchange
Gli scambi di chiave aggiuntivi vengono aggiunti alla proposal tramite \inlinecode{PQ\_KEM\_1}



Lo scambio \inlinecode{IKE\_FOLLOWUP\_KE} è introdotto specificatamente per trasferire dati sulla chiavi addizionali da realizzare in una CHILD SA.
In questo caso le chiavi aggiuntive vengono utilizzate per aggiornare il KEYMAT



\begin{itemize}
    \item flag \inlinecode{IKE\_FRAGMENTATION\_SUPPORT}: il peer dice di supportare la frammentazione IKEv2, affinchè venga utilizzata entrambi i peer devono supportarla.
    \item flag \inlinecode{INTERMEDIATE\_EXCHANGE\_SUPPORT}: il peer dice di supportare gli scambi intermedi
\end{itemize}

Una volta terminati gli scambi, per proteggere lo scambio \inlinecode{IKE\_AUTH} e gli scambi successivi vengno utilizzata le ultime chiavi calcolate
Dato che i dati trasferiti in questi scambi aggiuntivi vanno autenticati si aggiungono all'\inlinecode{AUTH} payload che poi andrà 



Il supporto per lo scambio aggiuntivo viene comunicato aggiungendo all'interno
dell \inlinecode{IKE\_SA\_INIT} il flag \inlinecode{IKE\_INT\_SUP} (che sta per
Intermediate Exchange Support).
Se anche il responder lo supporta lo includerà nel messaggio di risposta dello
scambio.




Considerazioni, L'IKE fragmentation viene introdotta a causa del NAT tuttavia nel nostro caso di satelliti non ha senso utilizzarla in quanto non credo che si utilizzi
il NAT soprattutto perchè introduce ritardi dovuti alla traduzione degli indirizzi



\section{Post-Quantum}

Un solo KEM con Kyber L1 usando come suite AES\_GCM has vabene come certificato dilithium L1

Nel KEM quanti cifrano?

Cioè l'initiator manda il certificato e poi il responder cifra 