% Chapter 1

\chapter{Introduction} 

\label{introduction}  

Introduzione al quantum computer e spiegare il nome, ovvero perchè si basa sui principi fisici.
Confronto con il computer classico utilizzato per calcolare la classe di complessità di un problema

L'attuale crittorafia a chiave pubblica è minacciata da due algoritmi pioneri in questo campo ovvero quello di Grover e Shor
Negli utilimi anni la minaccia del quantum computing, in particolare la loro potenza di calcolo
insieme agli algoritmi di Grover e shor ha 
ribaltato quelle che sono le carte in tavola, dato che consentono di risolvere in tempo
polinomiale i problemi su cui si basano gli schemi crittografici più diffusi 
tra cui il problema del logaritmi discreto alla base di DH e la fattorizzazione di numeri 
primi alla base di RSA.

Questo ha spinto a introdurre nuovi schemi di firma basati su problemi matematici più complessi, 
tra questi abbiamo i lattice-based, hash-based, ecc..
Recentemente tra quelli proposti ne sono stati standardizzati diversi, tra questi abbiamo:
kyber, dilithium, classim mceliece 

\section*{Criticità}

\begin{itemize}
	\item Tradeoff tra aumento della complessità con dimensioni chiave e velocità delle operazioni
	\item Requisiti della rete e in generale delle applicazioni
	\item Implementazioni 
	\item Affidabilità e dunque transizione da uno all'alto.
\end{itemize}


Ora il fatto che questi problemi si siano dimostrati computazionalmente onerosi non li rende 
ottimali anche per l'utilizzo pratico su tutte quelle che sono le infrastrutture di rete esistenti.

Questo perchè per contrastare l'incredibile potenza di calocolo del quantum computer occorre 
rendere il problema più complesso che in generale potrebbe portare e dimensioni delle chiavi molto 
grandi oppure ad operazioni di keygen, codifica e decoddifica molto lonte.

Dunque l'algoritmo dal punto di vista matematico soddisfa quelli che sono i requisiti 
tuttavia non è detta che soddisfi quelli che sono i requisiti che lo rendano adatto ad
essere applicato a contesti reali come quello dlele reti di computer.

Oltre ad un problema computazionale abbiamo anche un problema di fiducia dei cofronti di questi algoritmi, 
ovvero dato che sono stati appena introdotti l'implementazione potrebbe peccare da qualche punto di vista
Inoltre fare una transizione così drastica risulta molto problematico.






Io la metterei si dal punto di vista della ricerca ma anche dal punto di vista implementativo, ovvero 
non basta definire solamente nuovi schemi che dal punto di vista teorico possono essere sicuri 
ma questi devono poi trovare un'utilizzo pratico.

L'utilizzo pratico va in contro a diverse problematiche in particolare ha requisiti più stringenti 
che al momento della definizione matematiche dello schema non vengono presi in considerazione.
Si hanno constraint sia di usabilità che di fiducia nei loro confronti
poichè l'approccio standard è quello di aumentare le dimensioni delle chiavi in modo tale da contrapporti all'aumento di
capacità computazionale del quantum computer

Nel caso reale l'autemto di dimensione ha effetti significativi sulle prestazioni della rete
dato che possono portare a problematiche di frammentazione. E va considerata anche la latenza dovuta alle operazioni 
di cifratura e altre cose



\section*{Contributo Apportato}

L'obiettivo di questo lavoro è andare a vedere quali sono gli effetti di applicare primitive di questo 
tipo nei protocolli maggiormenti diffusi per la sicurezza delle comunicazioni.
In particolare considerando il caso specifico di comunicazioni satellitari, che hanno 
constraint importanti sul numero di pacchetti da scambiare e di conseguenza sulla dimesione di quest'utilimi

Una volta determinati quelli che sono gli effetti, siamo passati a verificare se il protocollo utilizzato 
rispetto alla sua applicazione, fosse quello ideale. 
In particolare dalle conclusioni del benchmarcking siamo arrivati ad una prima implementazione, molto spartana,
di quella che è una versione minimale di IKE.



\section*{Organizazione della Tesi}

Il proseguo della tesi sarà strutturato nel seguente modo:

\begin{itemize}
	\item Capitolo 2:  si danno le fondamenta matematica delle sicurezza nelle comunicazioni 
	sicure e di come queste vengono applicte nelle comunicazioni digitali. In particolare
	prendiamo in esame il caso di IPsec e di un suo protocollo ausiliario utilizzato per negoziarne 
	i parametri di sicurezza.
	\item Capitolo 3: descrizione di quello che è lo scenario applicativo che si prende in considerazione 
	
\end{itemize}

%----------------------------------------------------------------------------------------

% Define some commands to keep the formatting separated from the content 
\newcommand{\keyword}[1]{\textbf{#1}}
\newcommand{\tabhead}[1]{\textbf{#1}}
\newcommand{\code}[1]{\texttt{#1}}
\newcommand{\file}[1]{\texttt{\bfseries#1}}
\newcommand{\option}[1]{\texttt{\itshape#1}}

%----------------------------------------------------------------------------------------