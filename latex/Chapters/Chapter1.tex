% Chapter 1

\chapter{Introduction} 

\label{introduction}  

Introduzione al problema e al contensto in cui ci troviamo e le crtiticità presenti

Contributo apportato

%----------------------------------------------------------------------------------------

% Define some commands to keep the formatting separated from the content 
\newcommand{\keyword}[1]{\textbf{#1}}
\newcommand{\tabhead}[1]{\textbf{#1}}
\newcommand{\code}[1]{\texttt{#1}}
\newcommand{\file}[1]{\texttt{\bfseries#1}}
\newcommand{\option}[1]{\texttt{\itshape#1}}

%----------------------------------------------------------------------------------------

Fare un'introduzione al problema e dire quale è stato il contributo.\\
Suddividere il contributo in:

\begin{itemize}
	\item La parte di benchmarking, in cui andiamo a definire tutte quelle che sono le problematiche relative al post quantum 
	\item la parte di implementazione
\end{itemize}


\section{Post-Quantum}

Il postquantum può rompere gli schemi di crittografia classici tuttavia ì sono nati quelli nuovi.

\section{Problematiche}
Tuttavia il quantum computing porta ad aumentare la dimensione della chiave, in questo modo si perde di efficienza. Inoltre ha particolare effetto sui sistemi a chiave
pubblica, mentre su quelli basati su chiave segreta o funzioni di hash non sono molto vulnerabili a questo.
Le sfide della post-quantum cryptography sono le seguenti:

\begin{itemize}
	\item Occorre migliorare l'efficienza 
	\item Occorre migliorarne l'usabilità
\end{itemize}

Ovvero occorre preparare il mondo per una transizione alla crittografia post-quantum

Dire quelle che sono le problematiche del post quantum
Esempi di problematiche:
\begin{itemize}
\item non esiste il diffie hellman post quantum
\item i messaggi diventano molto più lunghi (problemi per frammentazione,
	certificati molto più grandi, chiavi pubbliche piu grandi)
\end{itemize}

Queste hanno conseguenze importanti sui protocolli che ne fanno uso a causa dell'aumento di dimensioni dellee chiavi.


Gli schemi sono divisi in 

- l1:aes128
- l3:aes192
- l5:aes256


