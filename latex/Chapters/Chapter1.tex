% Chapter 1

\chapter{State of The Art} 

\label{Chapter1}  

%----------------------------------------------------------------------------------------

% Define some commands to keep the formatting separated from the content 
\newcommand{\keyword}[1]{\textbf{#1}}
\newcommand{\tabhead}[1]{\textbf{#1}}
\newcommand{\code}[1]{\texttt{#1}}
\newcommand{\file}[1]{\texttt{\bfseries#1}}
\newcommand{\option}[1]{\texttt{\itshape#1}}

%----------------------------------------------------------------------------------------

\section{Problem}

Quando è stato annunciato che il quantum-computer avrebbe rotto algoritmi come RSA, ECDSA e DES Internet aveva ormai decretato la fine della crittografia.
Tuttavia c'è una sostanziale differenza tra rompere qualche crittosistema a rompere la crittografia, infatti esistono importanti classi di sistemi crittografici
oltre a quelli precedentemente citati. Tra questi abbiamo:

\begin{itemize}
	\item Hash-based cryptography
	\item Code-based cryptography
	\item Lattice-based cryptography
\end{itemize}

Tuttavia il quantum computing porta ad aumentare la dimensione della chiave, in questo modo si perde di efficienza. Inoltre ha particolare effetto sui sistemi a chiave
pubblica, mentre su quelli basati su chiave segreta o funzioni di hash non sono molto vulnerabili a questo.

Le sfide della post-quantum cryptography sono le seguenti:

\begin{itemize}
	\item Occorre migliorare l'efficienza 
	\item Occorre migliorarne l'usabilità
\end{itemize}

Ovvero occorre preparare il mondo per una transizione alla crittografia post-quantum

\subsection{Efficency}

Quanto deve essere la chiave per garantire lo stesso livello di sicurezza. Questa è necesssaria per i server in Internet che devono gestire centinaia di clienti ogni secondo.
Google con il suo motore di ricerca già esegue il redirect su http per non rallnetare

\subsection{Confidence}

La community deve avere confidenza nei sistemi e affinchè ciò sia possbibile è richiesto tempo affincheè i crittoanalisti cerchino degli attacchi per questi sistemi. 
Per esempio non esiste alcun problema studiato bene come il  Diffie-Hellman e la gente ha totale confidenza nei suoi confronti (da quando è stato introdotto non è mai stato rotto).


\subsection{Usability}



Dire quelle che sono le problematiche del post quantum
Esempi di problematiche:
\begin{itemize}
\item non esiste il diffie hellman post quantum
\item i messaggi diventano molto più lunghi (problemi per frammentazione,
	certificati molto più grandi, chiavi pubbliche piu grandi)
\end{itemize}

%----------------------------------------------------------------------------------------

\section{Solution}

Dare una panoramica delle attuali soluzioni tappabuchi che sono state
realizzate. Andare per stack network, per esempio TLS PQ, IKEv2 PQ

Dire quelle che sono le possibili soluzioni, in particolare perchè si è scelto
di utilizzare IKEv2 Post quantum

Parlare della libreria openquantumsafe

\subsection{TLS-PQ}

Dire che non è la cosa giusta da utilizzare in quanto non sono presenti RFC che
abbiano ben definito come realizzarla. Inoltre sono presenti online numerose
implementazioni custom di quest'ultima che tuttavia fanno uso sia di POst
quantum crittograpy che Quantum Key Distribution

\subsection{IKEv2-PQ}

Parlare nel dettaglio della sua architettura e altro nel secondo capitolo

Spiegare che ci concentreremo su questa per quanto riguarda le comunicazioni
sicure poichè è presente un RFC, c'èuna comunity ampia che si occupa di questo
problema


Test strongswan tempistiche, dimnesioni post quantum
Come peggiorano le prestazioni nel caso post quantum 

(Confronto tra i due casi)

Interessante vedere anche cosa cambia se utilizziamo scheduler real time

- Spinqs+ no perchè è una merda
- Guardare tutti gli algoritmi tranne quelli del 4° Round del NIST, l'unico da
provare è bike (HQC no)


Gli schemi sono divisi in 

- l1:aes128
- l3:aes192
- l5:aes256

Mentre per quanto riguarda i certificati vedre sia falcon che dilithium
(Falcon in teoria dovrebbe essere più lento a causa di operazioni float)
Sia in termini di dimensioni che di tempi



Provare solo quello che è L1 (livello di sicurezza 


\subsection{Liboqs}

Dire che in generale per la transizione al post quantum fanno tutti utilizzo di
questa libreria
